%%%%%%%%%%%%%%%%%%%%%%%%%%%%%%%%%%%%%%%%%%%%%%%%%%%%%%%%%%%%%%%%%%%%%%%%%%%%%%%
% Neuroimage-like layout
\documentclass[5p]{elsarticle}
\usepackage{amsmath,amsfonts,amssymb}
\usepackage{bm}
\usepackage{algorithm}
\usepackage{algorithmic}
\usepackage{url}
\usepackage[breaklinks=true,letterpaper=true,colorlinks,bookmarks=false]{hyperref}
\usepackage[table]{xcolor}

\definecolor{deep_blue}{rgb}{0,.2,.5}
\definecolor{dark_blue}{rgb}{0,.15,.5}

\hypersetup{pdftex,  % needed for pdflatex
  breaklinks=true,  % so long urls are correctly broken across lines
  colorlinks=true,
  linkcolor=dark_blue,
  citecolor=deep_blue,
}

% Float parameters, for more full pages.
\renewcommand{\topfraction}{0.9}        % max fraction of floats at top
\renewcommand{\bottomfraction}{0.8}     % max fraction of floats at bottom
\renewcommand{\textfraction}{0.07}      % allow minimal text w. figs
%   Parameters for FLOAT pages (not text pages):
\renewcommand{\floatpagefraction}{0.6}  % require fuller float pages
%    % N.B.: floatpagefraction MUST be less than topfraction !!


\def\B#1{\mathbf{#1}}
%\def\B#1{\bm{#1}}
\def\trans{^\mathsf{T}}
% A compact fraction
\def\slantfrac#1#2{\kern.1em^{#1}\kern-.1em/\kern-.1em_{#2}}

%%%%%%%%%%%%%%%%%%%%%%%%%%%%%%%%%%%%%%%%%%%%%%%%%%%%%%%%%%%%%%%%%%%%%%%%%%%%%%%%
\begin{document}

%\title{Comparing Connectomes Between Populations}
\title{Learning and comparing functional connectomes between populations}


\author[parietal,unicog,cea]{Ga\"el Varoquaux\corref{corresponding}}
\author[child_institute]{Cameron Craddock}

\cortext[corresponding]{Corresponding author}

\address[parietal]{Parietal project-team, INRIA Saclay-\^ile de France}
\address[unicog]{INSERM, U992}
\address[cea]{CEA/Neurospin b\^at 145, 91191 Gif-Sur-Yvette}
\address[child_institute]{Child Institute, New York}

\begin{abstract}
    We are the champion... of the world

    Scope: rest and task-based studies. But focusing on fMRI.
\end{abstract}

\begin{keyword}
    Functional connectivity, connectome, group study, effective
    connectivity, fMRI, resting-state
\end{keyword}

\maketitle
%%%%%%%%%%%%%%%%%%%%%%%%%%%%%%%%%%%%%%%%%%%%%%%%%%%%%%%%%%%%%%%%%%%%%%%%%%%%%%%%

\sloppy % Fed up with messed-up line breaks
\section{Introduction}

Review paper giving technical guidelines.

%%%%%%%%%%%%%%%%%%%%%%%%%%%%%%%%%%%%%%%%%%%%%%%%%%%%%%%%%%%%%%%%%%%%%%%%%%%%%%%

\section{Estimating functional connectomes}

%------------------------------------------------------------------------------
\subsection{Defining regions}

Different strategies: 
dense/sparse: covering a large fraction of the
brain or not == tradeoff between functional specificity and covering the
full brain
, hard vs soft boundaries.

\paragraph{Regions from atlases}

AAL \cite{tzourio-mazoyer2002a}, Harvard-oxford, sulci-based atlas.
Personal bias against the AAL, but note that it is widely used because of
SPM toolbox.

\paragraph{Defining regions from the literature}
Meta-analysis

\paragraph{FMRI-based function definition}
Distinction between activation and rest.
Activation: thresholding GLM maps, or using spheres around the activation
peak.
Rest: ICA-based approaches \cite{kiviniemi2009} \cite{shirer2012}
\cite{varoquaux2011}
Maybe use the ICA from the Smith 2009 paper

Clustering approaches, \cite{craddock2011}.

%------------------------------------------------------------------------------
\subsection{Estimating connections}

% Disclaimer: focus on correlation/second order statistics?

\paragraph{Signal extraction}

Once again separate task from rest (explain the difference between
\emph{ongoing} and \emph{evoked} activity at some point (intro?)

Task: beta time-series, and more recent stuff, \emph{e.g.} by Mumford and
Poldrack.

Rest: importance of confounds. Filtering + and detrending. Importance of
preprocessing.
Compcorr \cite{behzadi2007}
\cite{chai2011}
\cite{satterthwaite2012}

FMRI signals contributing to functional connectivity
have been found to lie in frequencies below 0.1 Hz \cite{cordes2000}
XXX: but cite recent work that shows that there is info at higher
frequency. 

\paragraph{Correlation and partial correlations}

Correlation, shrinkage of correlation \cite{ledoit2004,varoquaux2012},

Remark: difference between covariance and correlation matrix.

partial correlation, sparse iCov. \cite{smith2011,varoquaux2010b}. 

Give intuition on partial correlation/inverse covariance.

Remark that partial correlations consistently have no negative values.

Variety
of estimation strategies for sparse iCov, that have an impact on the
resulting network \cite{varoquaux2012}\footnote{Better estimation
algorithm is recent Honorio paper.}. Discuss setting the regularization
parameter.

%%%%%%%%%%%%%%%%%%%%%%%%%%%%%%%%%%%%%%%%%%%%%%%%%%%%%%%%%%%%%%%%%%%%%%%%%%%%%%%

\section{Comparing connections}

%------------------------------------------------------------------------------
\subsection{Mass-univariate approaches}

Boils down to a standard linear model => t-tests and co.
Stress that to have Gaussian-distributed correlations, a Fisher Z
transform is required.

Multiple comparison problem is quite an issue (scales a $p^2$) =>
permutation and max-T approach XXX: cite Ge
"resampling-based multiple testing for microarray data analysis", and
Nichols multiple comparison paper.

%------------------------------------------------------------------------------
\subsection{Modeling between-connection dependences}

Show that correlations co-fluctuation with images.

Ad-hoc model: network-based statistics \cite{zalesky2010}

Discuss the 'residual' strategy?
\cite{varoquaux2010b}

%%%%%%%%%%%%%%%%%%%%%%%%%%%%%%%%%%%%%%%%%%%%%%%%%%%%%%%%%%%%%%%%%%%%%%%%%%%%%%%

\section{Comparing Network Summary Statistics}

Small world networks, transport properties, resilient...

Warning note that correlation matrices have long-tailed degree
distribution by definition (cite recent Bullmore paper showing that).

%%%%%%%%%%%%%%%%%%%%%%%%%%%%%%%%%%%%%%%%%%%%%%%%%%%%%%%%%%%%%%%%%%%%%%%%%%%%%%%

\section{Predictive Modeling}

%%%%%%%%%%%%%%%%%%%%%%%%%%%%%%%%%%%%%%%%%%%%%%%%%%%%%%%%%%%%%%%%%%%%%%%%%%%%%%%

\section{Functional and effective connectivity}

%------------------------------------------------------------------------------
\subsection{From correlations to structural equation modeling}

\cite{mcintosh1994}
\cite{marrelec2007}
\cite{marrelec2009}

%------------------------------------------------------------------------------
\subsection{Matching model complexity to data}

Diatribe: all model are wrong, but some are useful

Setting the cursor between complex models based on a bio-physical
description, and simple phenomenological models such as correlation
matrices.

\cite{mcintosh2010}

%%%%%%%%%%%%%%%%%%%%%%%%%%%%%%%%%%%%%%%%%%%%%%%%%%%%%%%%%%%%%%%%%%%%%%%%%%%%%%%

\section{Conclusion}

{
%\clearpage
\section*{References} \small \bibliographystyle{elsarticle-num-names}
\bibliography{biblio} }

%%%%%%%%%%%%%%%%%%%%%%%%%%%%%%%%%%%%%%%%%%%%%%%%%%%%%%%%%%%%%%%%%%%%%%%%%%%%%%%


\end{document}
